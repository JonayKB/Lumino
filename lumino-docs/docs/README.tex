\documentclass[12pt,a4paper]{article}
\input{../../.common/preamble.sty}

\title{\textbf{Lumino Docs}}

\begin{document}
\maketitle

La documentación en un proyecto de software es un componente clave que asegura la comprensión, mantenimiento y evolución del sistema a lo largo de su ciclo de vida. Sirve como puente entre los equipos de desarrollo, los ``stakeholders'' y los usuarios finales, proporcionando una descripción clara de los objetivos, funcionalidades y arquitectura del proyecto. Sin una documentación adecuada, los equipos podrían enfrentarse a desafíos como malentendidos, errores en la implementación y dificultades para dar soporte o realizar actualizaciones futuras.\\

Existen muchas herramientas para documentar proyectos de software, pero dentro del mundo Python una de las más usadas es \href{https://www.mkdocs.org/}{MkDocs}. En realidad no es más que un generador estático pero que permite estructurar la documentación de forma muy clara y escribirla en formato Markdown.\\

Uno de los proyectos más interesantes que se han desarrollado sobre este ecosistema es \href{https://squidfunk.github.io/mkdocs-material/}{Material for MkDocs}. Tiene un aspecto visual muy potente y además ofrece gran cantidad de funcionalidades. Pero existe una secuela de este proyecto que es aún más interesante: \href{https://zensical.org/}{Zensical}.\\

\fbox{La idea es documentar el proyecto \textit{Lumino} (\texttt{pypas get lumino}) utilizando \textbf{Zensical}.}

\section{Contenido}

El grupo debe incluir en la documentación todos los puntos que se indican a continuación:

\subsection*{Introducción}

\begin{itemize}
    \item \textbf{Propósito del documento}: Explicar los objetivos y el alcance de la documentación.
    \item \textbf{Visión general del proyecto}: Breve descripción del software, su objetivo principal y el problema que soluciona.
    \item \textbf{Audiencia objetivo}: Definir quiénes usarán esta documentación (desarrolladores, testers, usuarios finales, etc.).
\end{itemize}

\subsection*{Requisitos del proyecto}

\begin{itemize}
    \item \textbf{Requisitos funcionales}: Funcionalidades clave que el sistema debe cumplir.
    \item \textbf{Requisitos no funcionales}: Rendimiento, seguridad, escalabilidad, usabilidad, etc.
    \item \textbf{Restricciones}: Factores técnicos, legales o de negocio que afectan el desarrollo.
    \item \textbf{Casos de uso}: Diagramas o descripciones de los escenarios de interacción con el sistema.
\end{itemize}

\subsection*{Diseño del sistema}

\begin{itemize}
    \item \textbf{Arquitectura del sistema}: Diagrama y descripción de los componentes principales.
    \item \textbf{Modelo de datos}: Esquema de base de datos y relaciones entre entidades.
    \item \textbf{Diagramas}: Diagramas de clases, de secuencia, de actividades...
    \item \textbf{Decisiones de diseño}: Justificación de las tecnologías, patrones y enfoques elegidos.
\end{itemize}

\subsection*{Implementación}

\begin{itemize}
    \item \textbf{Estructura del código}: Organización del repositorio y convenciones usadas.
    \item \textbf{Tecnologías y herramientas}: Lenguajes, frameworks, bibliotecas, y entornos de desarrollo.
    \item \textbf{Instrucciones de configuración}: Pasos para instalar dependencias y configurar el entorno.    
\end{itemize}

\subsection*{Pruebas}

\begin{itemize}
    \item \textbf{Estrategia de pruebas}: Métodos utilizados (unitarias, de integración, de aceptación).
    \item \textbf{Casos de prueba}: Ejemplos específicos y sus resultados esperados.
    \item \textbf{Cobertura de pruebas}: Indicadores de qué partes del código están siendo probadas.
    \item \textbf{Automatización}: Herramientas y scripts utilizados para pruebas automáticas.
\end{itemize}

\subsection*{Despliegue}

\begin{itemize}
    \item \textbf{Entorno de producción}: Descripción de los servidores, servicios y configuración necesaria.
    \item \textbf{Proceso de despliegue}: Pasos detallados para implementar el software.
    \item \textbf{Planes de recuperación}: Estrategias para manejar errores durante el despliegue.
\end{itemize}

\subsection*{Manuales de usuario}

\begin{itemize}
    \item \textbf{Instrucciones básicas}: Cómo instalar, configurar y usar el software.
    \item \textbf{Guías avanzadas}: Funcionalidades específicas o tareas complejas.
    \item \textbf{Solución de problemas}: Respuestas a preguntas frecuentes y resolución de errores comunes.
\end{itemize}

\subsection*{Mantenimiento y actualización}

\begin{itemize}
    \item \textbf{Política de mantenimiento}: Frecuencia de actualizaciones y soporte.
    \item \textbf{Registro de cambios (Changelog)}: Historial de versiones y mejoras realizadas.
    \item \textbf{Plan de escalabilidad}: Posibles mejoras para soportar más usuarios o nuevas funcionalidades.
\end{itemize}

\subsection*{Conclusiones y futuro}

\begin{itemize}
    \item \textbf{Estado actual del proyecto}: Resumen del progreso y alcance cumplido.
    \item \textbf{Futuras actualizaciones}: Funcionalidades o mejoras planificadas.
    \item \textbf{Lecciones aprendidas}: Reflexión sobre el desarrollo y recomendaciones para futuros proyectos.
\end{itemize}

\section{Requerimientos}

Para que puedas trabajar con normalidad en este ejercicio debes tener instalado en tu máquina:

\begin{enumerate}
    \item \href{https://docs.astral.sh/uv/}{\texttt{uv}} $\rightarrow$ Gestor de paquetería y proyectos Python.
    \item \href{https://just.systems/man/en/}{\texttt{just}} $\rightarrow$ Lanzador de comandos como recetas.
\end{enumerate}

\section{Puesta en marcha}

Se proporciona una \textit{receta} \href{https://just.systems/man/en/}{\texttt{just}} para la puesta en marcha del proyecto:

\begin{minted}[
    numbersep=5pt,
    frame=single,
    framesep=3mm,
]{bash}
just setup
\end{minted}

¿Qué ha ocurrido?

\begin{itemize}
    \item Se ha creado un entorno virtual en la carpeta \verb|.venv|
    \item Se han instalado las dependencias del proyecto.
\end{itemize}

En la estructura del proyecto es importante identificar lo siguiente:\\

\begin{tabular}{|l|l|}
    \hline
    \texttt{docs/} & Carpeta con ficheros \textit{Markdown} para documentar el proyecto\\
    \hline
    \texttt{zensical.toml} & Fichero de configuración principal del proyecto \textit{Zensical}\\
    \hline
\end{tabular}

\section{Flujo de trabajo}

Para trabajar con el proyecto habrá que ir incorporando contenido en \verb|docs/| y configuraciones en \verb|zensical.toml|.\\

Se proporciona una \textit{receta} \href{https://just.systems/man/en/}{\texttt{just}} para \textbf{renderizar el proyecto en desarrollo}:

\begin{minted}[
    numbersep=5pt,
    frame=single,
    framesep=3mm,
]{bash}
just runserver
\end{minted}

La receta anterior también se puede lanzar directamente con \verb|just| ya que es la receta \textit{por defecto}. Podrás visualizar el proyecto en tu navegador accediendo a: \verb|http://localhost:8000|\\

Se proporciona una \textit{receta} \href{https://just.systems/man/en/}{\texttt{just}} para \textbf{construir el proyecto}:

\begin{minted}[
    numbersep=5pt,
    frame=single,
    framesep=3mm,
]{bash}
just build
\end{minted}

La receta anterior crea una carpeta \verb|site/| donde estarán los ficheros estáticos con el proyecto generado.

\end{document}
